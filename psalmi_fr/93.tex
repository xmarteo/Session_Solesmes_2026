\item Dieu qui fais justice, Seigneur, * Dieu qui fais justice, parais !
\item Lève-toi, juge de la terre ; * aux orgueilleux, rends ce qu'ils méritent.
\item Combien de temps les impies, Seigneur, * combien de temps vont-ils triompher ?
\item Ils parlent haut, ils profèrent l'insolence, * ils se vantent, tous ces malfaisants.
\item C'est ton peuple, Seigneur, qu'ils piétinent, * et ton domaine qu'ils écrasent ;
\item Ils massacrent la veuve et l'étranger, * ils assassinent l'orphelin.
\item Ils disent : « Le Seigneur ne voit pas, * le Dieu de Jacob ne sait pas ! »
\item Sachez-le, esprits vraiment stupides ; * insensés, comprendrez-vous un jour ?
\item Lui qui forma l'oreille, il n'entendrait pas ? † il a façonné l'œil, et il ne verrait pas ? * il a puni des peuples et ne châtierait plus ? 
\item Lui qui donne aux hommes la connaissance, † il connaît les pensées de l'homme, * et qu'elles sont du vent !
\item Heureux l'homme que tu châties, Seigneur, * celui que tu enseignes par ta loi,
\item pour le garder en paix aux jours de malheur, * tandis que se creuse la fosse de l'impie.
\item Car le Seigneur ne délaisse pas son peuple, * il n'abandonne pas son domaine :
\item on jugera de nouveau selon la justice ; * tous les hommes droits applaudiront.
\item Qui se lèvera pour me défendre des méchants ? * Qui m'assistera face aux criminels ?
\item Si le Seigneur ne m'avait secouru, * j'allais habiter le silence.
\item Quand je dis : « Mon pied trébuche ! » * ton amour, Seigneur, me soutient.
\item Quand d'innombrables soucis m'envahissent, * tu me réconfortes et me consoles.
\item Es-tu l'allié d'un pouvoir corrompu * qui engendre la misère au mépris des lois ?
\item On s'attaque à la vie de l'innocent, * le juste que l'on tue est déclaré coupable.
\item Mais le Seigneur était ma forteresse, * et Dieu, le rocher de mon refuge.
\item Il retourne sur eux leur méfait : + pour leur malice, qu'il les réduise au silence, * qu'il les réduise au silence, le Seigneur notre Dieu.
